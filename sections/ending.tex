\section{ZAKOŃCZENIE}

\subsection{Podsumowanie}

Utworzona aplikacja spełnia założenia projektowe. Udało się przygotować system,
który pozwala na wykonywanie dowolnych działań na bazie danych za pomocą
natywnych zapytań SQL. Program udostępnia administratorom funkcje niedostępne w
innych systemach CMS i headless CMS dostępnych na rynku.

\subsection{Możliwości rozwoju}

Aplikacja posiada większość funkcji typowego systemu CMS, jak tworzenie nowego
typu danych i zarządzanie danymi za pomocą formularzy, a nie przez pisanie SQL.
Brakuje jednak możliwości zmiany struktury tabel za pomocą formularza.
Niewygodne jest również dodawanie kluczy obcych, które wymaga pamiętania nazwy
tabeli, do której chce się odnieść oraz nazwy kolumny przechowującej klucze
główne.

Funkcją dostępną w typowych systemach CMS, której zupełnie brakuje w programie
jest zarządzanie mediami. Należałoby się zastanowić, jak połączyć system,
którego głównym sposobem zarządzania jest pisanie natywnych zapytań SQL z
systemem zarządzania mediami.

Można by użyć do tego celu jednego z typów danych bazy Postgres przeznaczonego do
przechowywania danych binarnych. Popularnym rozwiązaniem jest przechowywanie
danych binarnych w systemie plików i przechowywanie nazwy pliku w bazie danych.
Takie rozwiązanie utrudniłoby jednak administratorom zarządzanie danymi. W celu
ustalenia optymalnego rozwiązania problemu, wymagane są dalsze badania.

Kolejną funkcją, jaką warto w przyszłości zaimplementować jest udostępnienie
użytkownikom nie będącym administratorami okrojonego panelu administratora.
Można by użyć do tego komponentów z panelu administratora, lub stworzyć
niezależny panel, który zawiera interfejs przeznaczony do wywoływania punktów
końcowych przygotowanych przez administratorów. Zaimplementowanie obydwu
pomysłów stworzyłoby interfejs nie gorszy, niż w innych popularnych systemach
CMS.

Kolejną funkcją, jaką warto w przyszłości zaimplementować jest zwracanie danych
przy wywołaniu punktu końcowego nie jako dokument json, lecz jako szablon HTML
wypełniony danymi wynikowymi. Taka funkcja byłaby prosta w implementacji.
Najtrudniejszą częścią implementacji byłoby przygotowanie interfejsu zarządzania
szablonami w panelu administratora.
