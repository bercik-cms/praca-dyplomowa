\section{ZAŁOŻENIA PROJEKTOWE TWORZONEGO SYSTEMU}

\subsection{Założenia funkcjonalne}

\begin{enumerate}

    \item Responsywny interfejs aplikacji.

    \item System kart pozwalający na wygodną pracę z wieloma funkcjami
        systemu w tym samym czasie.

    \item Interfejs umożliwiający tworzenie nowych tabel w bazie danych bez
        pisania zapytań SQL.

    \item Interfejs umożliwiający proste zarządzanie danymi w bazie danych bez
        pisania zapytań SQL.

    \item Edytor zapytań SQL wykonywanych natychmiast.

    \item Możliwość wygodnego pisania dowolnej ilości zapytań SQL w edytorze
        zapytań wykonywanych natychmiast.

    \item Możliwość manipulowania listą zapytań SQL w edytorze zapytań
        wykonywanych natychmiast. W tym możliwość zmiany kolejności zapytań,
        możliwość wyłączania zapytania z uruchomienia i możliwość usuwania
        zapytania z listy zapytań.

    \item Możliwość testowego wykonywania zapytań bez zmiany danych w bazie
        danych.

    \item Możliwość podglądu wyników zapytań zwracających dane przy testowym
        wykonywaniu listy zapytań.

    \item Możliwość napisania zapytania testowego, które zostanie wykonane przed
        i po wykonaniu listy zapytań w celu porównania wyniku zapytania
        testowego.

    \item Możliwość wygodnego porównania wyniku zapytania testowego przed i po
        wykonaniu listy zapytań.

    \item Możliwość wygenerowania diagramu ER bazy danych.

    \item Możliwość wygenerowania diagramu przed i po testowym wykonaniu listy
        zapytań SQL w celu porównania struktury bazy danych.

    \item Możliwość pełnoekranowego podglądu diagramu ER.

    \item Możliwość tworzenia przez administratora punktów końcowych, których
        wywołanie za pomocą zapytania HTTP spowoduje wykonanie zdefiniowanego
        drzewa zapytań SQL i zwrócenie wyników zapytań w odpowiedzi HTTP.

    \item Możliwość definiowania zapytań podrzędnych do zapytania z drzewa
        zapytań SQL, które zostaną wykonane dla każdego rzędu tablicy będącej
        wynikiem wykonania drzewa nadrzędnego.

    \item Możliwość korzystania z danych z zapytań nadrzędnych w zapytaniach
        podrzędnych w drzewie zapytań SQL.

    \item Możliwość ustalenia adresu URL tworzonego punktu końcowego.

    \item Możliwość ustalenia metod HTTP, których można użyć do wywołania punktu
        końcowego.

    \item Możliwość korzystania z danych przychodzących w zapytaniu HTTP w
        zapytaniach SQL wykonywanych przy wywołaniu punktu końcowego.

    \item Możliwość ograniczenia możliwości wykonania punktu końcowego do
        ustalonej listy grup użytkowników.

    \item Możliwość edycji i usuwania istniejących punktów końcowych.

    \item Możliwość tworzenia kont użytkowników przez administratora.

    \item Możliwość tworzenia dowolnej ilości kont użytkowników w ramach jednej
        operacji.

    \item Możliwość przypisywania tworzonych kont użytkowników do grup
        użytkowników.

    \item Możliwość usuwania kont użytkowników przez administratora.

    \item Możliwość zmiany hasła przez użytkowników.

    \item Przechowywanie danych użytkowników w systemie w taki sposób, że dane
        przechowywane w pamięci nieulotnej nie pozwolą na zalogowanie się do
        systemu.

\end{enumerate}

\subsection{Założenia niefunkcjonalne}

\begin{enumerate}

    \item Wielowątkowe działanie aplikacji backend.

    \item Asynchroniczne wykonywanie operacji wejścia-wyjścia przez aplikację
        backend.

    \item Wykorzystywanie mniej niż dziesięciu megabajtów pamięci RAM przez
        aplikację backend w stanie spoczynku.

\end{enumerate}
