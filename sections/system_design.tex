\section{PROJEKT SYSTEMU}

\subsection{Projekt aplikacji frontend}

Aplikacja frontend została zaprojektowana w sposób typowy dla aplikacji SPA
\cite{mikowski2013single} z wykorzystaniem biblioteki React.js. Aplikacja nie
wymaga ładowania nowej strony przez przeglądarkę podczas pracy.

Składa się z komponentów, z których niektóre są używane w wielu miejscach
aplikacji. Komponenty, które implementują skomplikowaną funkcjonalność zawierają
komponenty podrzędne.

Główny komponent aplikacji to widok kart (\verb|TabsView|). Komponent zawiera
pasek kart (\verb|TabBar|) oraz komponent służący do wybierania edytorów
(\verb|EditorSelector|). Komponent \verb|TabBar| implementuje listę kart, która
pozwala użytkownikowi na przełączanie się pomiędzy edytorami. Karty, które nie
są aktywne są chowane za pomocą ustawienia wartości CSS \verb|display| na
\verb|none|. W ten sposób, nie tracą danych przechowywanych w stanie komponentu
przy przełączaniu kart. Komponent \verb|EditorSelector| pozwala na wybór
edytora. Edytory zawierają interfejs, przy użyciu którego administrator wykonuje
pracę na systemie CMS.

Komponent \verb|EditorSelector| pobiera dane o dostępnych edytorach ze struktury
danych. Umożliwia to łatwe dodawanie nowych edytorów do aplikacji. Strukturę
danych widać na listingu \ref{editorListListing}. Struktura danych przechowuje
dane o folderach, które mogą zawierać edytory. Foldery i edytory mają ikonę oraz
opis, a edytory mają również komponent. Przy wyborze edytora, komponent
przypisany do tego edytora zostanie wyświetlony na ekranie.

\lstinputlisting[
    float=h!,
    frame=tb,
    label={editorListListing},
    caption={Struktura danych przechowująca dane o edytorach}
]{./code/editors.ts}

Aplikacja frontend pozwala użytkownikowi na zmianę motywu. Użytkownik może
wybrać motyw jasny i ciemny. Przełączenie motywu powoduje podmienienie zmiennych
CSS z kodu aplikacji. Wybrany przez użytkownika motyw jest zapisywany z
wykorzystaniem standardowej funkcji przeglądarki \verb|localStorage| i ładowany
przy uruchomieniu aplikacji. Domyślny motyw to motyw jasny (ciemny tekst na
jasnym tle). Kod odpowiedzialny za przełączanie motywu został zamieszczony na listingu
\ref{schemeTogglingListing}.

\lstinputlisting[
    float=h!,
    frame=tb,
    label={schemeTogglingListing},
    caption={Kod odpowiedzialny za przełączanie motywu}
]{./code/schemeToggling.ts}

\subsection{Projekt aplikacji backend}

Aplikacja backend nie implementuje żadnego z wzorców architektonicznych. Cała
aplikacja jest podzielona na trzy moduły:

\begin{itemize}
    \item Moduł zawierający algorytmy.
    \item Moduł zawierający funkcje obsługujące zapytania HTTP.
    \item Moduł zawierający serwisy.
\end{itemize}

Moduł zawierający algorytmy zawiera najbardziej skomplikowane logicznie elementy
programu. Znajdują się tutaj funkcje:

\begin{itemize}
    \item generujące diagram ER w składni mermaid.js z tabel zawartych w
        systemie,
    \item parsujące zapytania SQL zawierające odniesienia do zmiennych,
    \item wykonywujące sparsowane drzewo zapytań SQL zawierających odniesienia
        do zmiennych.
\end{itemize}

Serwisy zawierają logikę biznesową i wywołują algorytmy. Duża część funkcji
zawartych w module serwisów odpowiada za zbieranie danych o tabelach zawartych w
bazie danych.

Funkcje obsługujące zapytania HTTP zawierają ograniczoną logikę. Funkcje te
powinny jedynie wywoływać serwisy i przekształcać możliwe błędy na odpowiedzi
HTTP.
