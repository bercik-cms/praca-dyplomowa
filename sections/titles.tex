\begin{center}
    \arialsection\fontsize{14pt}{14pt}\selectfont\bfseries
    Opracowanie i implementacja systemu Headless CMS z dodatkową możliwością modelowania warstwy
    danych przy pomocy natywnego SQLa
\end{center}

{\arialsection\fontsize{14pt}{14pt}\selectfont\bfseries \noindent Streszczenie}

\vspace{.5cm}

Systemy CMS oraz Headless CMS mają tendencję do ograniczania możliwości
operowania na danych do statycznych operacji zdefiniowanych przez system. Może
to powodować trudności, gdy wymagane jest wykonanie skomplikowanych operacji na
bazie danych. Ponadto, wysyłanie zapytań do bazy danych, z której korzysta
system CMS jest często odradzanie i niewspierane przez twórców systemu CMS.
Celem pracy dyplomowej jest napisanie systemu Headless CMS, który nie ogranicza
możliwości manipulacji bazą danych. Przygotowany system CMS osiąga ten cel,
przez umożliwienie administratorom wykonywania natywnych zapytań SQL na bazie
danych, z której korzysta system.  Kolejną kluczową funkcją jest pozwolenie
administratorom na definiowanie zapytań, które zostaną wykonane przy zapytaniu
HTTP do systemu CMS, i których dane wynikowe zostaną zwrócone w odpowiedzi HTTP.

\vspace{.5cm}

\noindent Słowa kluczowe: CMS, Headless CMS, bazy danych, SQL, zarządzanie treścią,
warstwa danych

\vspace{1cm}

\begin{center}
    \arialsection\fontsize{14pt}{14pt}\selectfont\bfseries
    The development and the implementation of a Headless CMS system with support of modelling of the
    data layer by native SQL queries
\end{center}

{\arialsection\fontsize{14pt}{14pt}\selectfont\bfseries \noindent Summary}

\vspace{.5cm}

CMS and Headless CMS systems have a tendency to limit the ability to operate on
data to static operations defined by the system. It can cause difficulties when
it's necessary to execute complex operations on the database. Furthermore,
sending queries to the database used by the CMS system is often discouraged and
not supported by the authors of the CMS system. The aim of the thesis is to
prepare a Headless CMS system, which does not limit the ability of manipulating
the database. The aim is achieved by allowing the administrators to execute
native SQL queries on the database used by the system. Another key feature is to
let administrators define queries which will be executed after an HTTP request
to the CMS system and the resulting data will be returned in the HTTP response.

\vspace{.5cm}

\noindent Keywords: CMS, Headless CMS, databases, SQL, content management,
data layer

\afterpage{\null\newpage}
\clearpage
