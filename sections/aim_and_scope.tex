\section{CEL I ZAKRES PRACY}

\subsection{Cel pracy}

Celem pracy było przygotowanie systemu CMS. System powinien zawierać większość
funkcji dostępnych w typowym systemie CMS, jak tworzenie nowych typów danych i
manipulacja danymi znajdującymi się w systemie, jednak nie są to główne funkcje
systemu. Głównym założeniem przy projektowaniu i implementacji było stworzenie
sytemu, który nie ogranicza możliwości operowania danymi przez administratora.

Główny cel został osiągnięty przez pozwolenie administratorowi na wykonywanie
natywnych zapytań SQL na bazie danych. Zaimplementowano aplikacje frontend i
backend, które razem umożliwiają administratorowi wygodne pisanie zapytań SQL i
testowanie ich wpływu na bazę danych.

Nie mniej ważnym założeniem było umożliwienie administratorowi definiowania
punktów końcowych, których wywołanie zapytaniem HTTP powoduje wykonanie
zdefiniowanego przez administratora drzewa zapytań SQL. Przygotowany edytor
pozwala na modelowanie struktury danych, które zostaną zwrócone w odpowiedzi
HTTP.

\subsection{Zakres funkcji systemu}

Program przygotowany w ramach pracy można podzielić na dwie części: frontend i
backend:

\begin{enumerate}
    \item Frontend.

    \begin{itemize}

        \item Aplikacja SPA wykorzystującą bibliotekę react.js. Strona
        porozumiewa się z serwerem za pomocą API REST.
        
        \item Jest napisany w języku Typescript i wykorzystuje system typów w
        każdym miejscu, gdzie jest to możliwe.

        \item Strona jest responsywna dzięki zastosowanym nowoczesnym opcjom CSS
        - grid i flexbox. Strona korzysta ze zmiennych CSS. Zmienne CSS
        umożliwiły łatwą implementację funkcji zmiany motywu aplikacji przez
        użytkownika.

        \item Implementuje ważniejsze funkcje typowego panelu administratora
        systemu zarządzania treścią: tworzenie nowych tabel, zarządzanie danymi
        w tabeli.

        \item Implementuje zaawansowany edytor SQL z podświetlaniem składni,
        funkcją zmiany kolejności zapytań, funkcją tymczasowego wyłączania
        zapytania. Wyświetla wyniki każdego zapytania pod kodem zapytania.
        Pozwala pisać zapytanie testowe, którego wynik zostanie pobrany przed
        wykonaniem listy zapytań jak i po wykonaniu. Pozwala na wygenerowanie
        diagramu ER przed i po wykonaniu listy zapytań. Pozwala na testowanie
        listy zapytań z pomocą transakcji, która zostanie wycofana przed zwrotem
        danych.

        \item Implementuje edytor punktów końcowych pozwalający na pisanie
        złożonych drzew zapytań, w których zapytania podrzędne mają dostęp do
        danych wynikających z wykonania zapytań nadrzędnych. Administrator może
        testować punkt końcowy przed wdrożeniem z wykorzystaniem edytora danych
        testowych. Administrator może ograniczać możliwość wywołania punktu
        końcowego do listy grup użytkowników.

        \item Implementuje interfejs tworzenia użytkowników.

    \end{itemize}

    \item Backend.

    \begin{itemize}

        \item Backend aplikacji został napisany w języku rust. Jest
        asynchroniczny, co pozwala na obsługiwanie wielu zapytań na mniejszej
        ilości wątków. Wykorzystuje asynchroniczny runtime Hyper.

        \item Współpracuje z bazą danych PostgreSQL. Używa tabel
        zawierających dane o tabelach w bazie danych, przez co nie musi sam
        przechowywać metadanych o tabelach.

        \item Korzysta z transakcji. Są używane do testowania pisanych
        przez administratora zapytań SQL.

        \item Implementuje generowanie diagramu ER w składni mermaid.js z danych
        pobranych z tabel zawierających metadane o tabelach w bazie danych.

        \item Implementuje bezpieczne wykonywanie drzewa zapytań SQL z użyciem
        niezaufanych danych z przychodzącego zapytania HTTP. Zapytania podrzędne
        mają dostęp do zapytań nadrzędnych. Każde wykorzystanie danych w
        zapytaniach jest zabezpieczone przed atakami SQL injection.

        \item Implementuje system użytkowników z wykorzystaniem technologii json
        web token. Bezpiecznie przechowuje hasła użytkowników za pomocą
        algorytmu Argon2.

    \end{itemize}

\end{enumerate}
