\section{WSTĘP}

W ostatnich latach można zaobserwować zanikanie podziału pomiędzy mediami
drukowanymi a internetowymi. Większość organizacji publikujących treści prowadzi
strony internetowe. Wiele innych organizacji nie zajmujących się mediami zmaga
się z problemem zarządzania treścią, którą produkują \cite{Mauthe_2004}. 

Programy, które wspomagają tworzenie i zarządzanie treścią nazywa się systemami
zarządzania treścią (ang. content management system — CMS). Do tych aplikacji
zaliczają się dość proste aplikacje służące do zarządzania plikami, jak i
skomplikowane systemy obsługujące wiele rodzajów danych i urządzeń.

Pośród organizacji, które korzystają z systemów zarządzania treścią są między
innymi: CNN, New York Times, Fox News, Wall Street Journal, Armia Stanów
Zjednoczonych. Wszystkie te organizacje korzystają z sytemu Wordpress. Ponadto,
oprócz dużych organizacji, serwis wordpress.com w roku 2009 udostępniał usługi
ponad trzem i pół miliona blogom \cite{brazell2010wordpress}. Mimo to, nie
powstał jeszcze uniwersalny system CMS, który spełniłby zapotrzebowania zarówno
małych instytucji, jak i dużych organizacji, które przechowują duże ilości
danych. Wątpliwe jest nawet, czy taki system mógłby zostać zbudowany, ponieważ
wymagania z różnych przypadków użycia są zupełnie inne \cite{Mauthe_2004}.

Ostatnio, popularne stały się systemy headless CMS \cite{WhyHeadlessCMS}. Są to
systemy CMS pozbawione warstwy prezentacji. Są one zazwyczaj mniej skomplikowane
od tradycyjnych systemów i nie ograniczają wyboru technologii, której można użyć
do implementacji niestandardowej warstwy prezentacji. Systemy te dobrze
spełniają się w architekturze mikroserwisów, lub gdy produkowane treści są
wyświetlane w różnych formach — na przykład na stronie internetowej i w
aplikacji mobilnej \cite{WhyHeadlessCMS}.

\medspace

Zdaniem autora pracy, mimo że systemy headless CMS nie ograniczają warstwy
prezentacji, nie oferują większych, niż tradycyjne systemy możliwości operowania
na warstwie danych. Z tego powodu, w ramach tej pracy, przygotowano system
headless CMS, którego założeniem było nie ograniczanie możliwości operowania na
bazie danych przez umożliwienie administratorom wykonywania natywnych zapytań
SQL.

Kolejnym punktem, gdzie ograniczane są możliwości administratora jest
definiowanie własnych punktów końcowych. Większość systemów CMS zupełnie nie
posiada takiej funkcji, lub wymaga ona pisania przez administratora kodu w
języku innym, niż SQL. Z tego powodu, kolejnym założeniem przygotowanego systemu
było umożliwienie administratorom definiowania własnych punktów końcowych HTTP i
zapytań SQL, które zostaną wykonane przy wywołaniu punktów końcowych i których
wyniki zostaną zwrócone w odpowiedzi HTTP.

\vspace{2cm}

W rozdziale 2 przedstawione zostały systemy CMS jak i inne systemy spełniające
pewien zakres problemów, który chce spełnić przygotowany w ramach pracy system.
W celu uzasadnienia potrzeby przygotowania systemu, zwrócono uwagę na problemy,
które nie są rozwiązane przez poszczególne systemy dostępne na rynku. W
rozdziale 3 przedstawiono założenia projektowe funkcjonalne i niefunkcjonalne
tworzonego systemu. W rozdziale 4 omówiono szczegółowo, jak zaimplementowano
każdą z kluczowych funkcji systemu. W rozdziale 5 omówiono używane przy pisaniu
programu sposoby weryfikacji działania kodu. Pracę kończy rozdział z
podsumowaniem i planem dalszego rozwoju.
